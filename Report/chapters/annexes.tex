% Annexes
\appendix
\chapter{Annexes}

\section{Configuration Docker pour Redis}
\label{ann:docker_redis}
\begin{lstlisting}[language=Python, caption=Fichier de configuration Docker pour Redis, label=lst:docker_redis]
redis:
  build:
    context: ./.docker/redis/.
    dockerfile: Dockerfile.redis
  ports:
    - "${REDIS_PORT}:${REDIS_PORT}"
  volumes:
    - ./.docker-data/redis/data:/data
    - ./.docker-data/redis/logs:/var/log/redis
  restart: always
  env_file:
    - ./.docker/redis/.env.redis
  environment:
    - REDIS_ARGS=--requirepass ${REDIS_PASSWORD}
  healthcheck:
    test: ["CMD", "redis-cli", "-a", "${REDIS_PASSWORD}", "ping"]
    interval: 30s
    timeout: 10s
    retries: 5
\end{lstlisting}
\section{Configuration Docker pour MongoDB}
\label{ann:docker_mongo}
\begin{lstlisting}[language=Python, caption=Fichier de configuration Docker pour MongoDB, label=lst:docker_mongo]
# MongoDB Primary
mongo1:
  build:
    context: ./.docker/mongo/
    dockerfile: Dockerfile.mongo
  ports:
    - 27017:27017
  restart: always
  volumes:
    - ./.docker-data/mongo1/data:/data/db
  env_file:
    - ./.docker/mongo/.env.mongo
  healthcheck:
    test: echo 'db.runCommand("ping").ok' | mongosh localhost:27017/test --quiet
    interval: 30s
    timeout: 10s
    retries: 5
    start_period: 20s

# MongoDB Replica Set
mongo2:
  build:
    context: ./.docker/mongo/
    dockerfile: Dockerfile.mongo
  ports:
    - 27018:27017
  restart: always
  volumes:
    - ./.docker-data/mongo2/data:/data/db
  env_file:
    - ./.docker/mongo/.env.mongo
  healthcheck:
    test: echo 'db.runCommand("ping").ok' | mongosh localhost:27018/test --quiet
    interval: 30s
    timeout: 10s
    retries: 5
    start_period: 20s

# MongoDB Replica Set
mongo3:
  build:
    context: ./.docker/mongo/
    dockerfile: Dockerfile.mongo
  ports:
    - 27019:27017
  restart: always
  volumes:
    - ./.docker-data/mongo3/data:/data/db
  env_file:
    - ./.docker/mongo/.env.mongo
  healthcheck:
    test: echo 'db.runCommand("ping").ok' | mongosh localhost:27019/test --quiet
    interval: 30s
    timeout: 10s
    retries: 5
    start_period: 20s
\end{lstlisting}

\section{Replica Set MongoDB}
\label{ann:replica_set}
\begin{lstlisting}[language=Python, caption=Replica Set MongoDB, label=lst:replica_set]
#!/bin/bash

echo 'Waiting for MongoDB to start...'
until mongosh --eval "db.adminCommand('ping')" --authenticationDatabase admin --username ${MONGO_INITDB_ROOT_USERNAME} --password ${MONGO_INITDB_ROOT_PASSWORD}; do
  >&2 echo 'MongoDB is unavailable - sleeping'
  sleep 5
done

echo 'Initializing replica set for Docker network...'
mongosh -u ${MONGO_INITDB_ROOT_USERNAME} -p ${MONGO_INITDB_ROOT_PASSWORD} --authenticationDatabase admin --eval "rs.initiate({_id: 'rs0', members: [{ _id: 0, host: 'mongo1:27017' }, { _id: 1, host: 'mongo2:27017' }, { _id: 2, host: 'mongo3:27017' }]})"

echo 'Replica set initialized.'
\end{lstlisting}

\section{Configuration Docker pour Python}
\label{ann:docker_poetry}
\begin{lstlisting}[language=Python, caption=Fichier de configuration Docker pour Python, label=lst:docker_poetry]
poetry:
  build:
    context: ./.docker/poetry/.
    dockerfile: Dockerfile.poetry
  stdin_open: true
  tty: true
  volumes:
    - .:/workspace
  working_dir: /workspace
  environment:
    - POETRY_VIRTUALENVS_IN_PROJECT=true
  command: [ "/bin/sh" ]
\end{lstlisting}

\section{Création des Documents JSON}
\begin{lstlisting}[language=Python, caption=Script de création des documents JSON, label=lst:code_json]
  import json
  import os
  
  # Table de correspondance des colonnes
  tableCorespondance = {
    "AVIONS.txt": ["NumAv", "NomAv", "CapAv", "VilleAv"],
    "CLIENTS.txt": ["NumCl", "NomCl", "NumRueCl", "NomRueCl", "CodePosteCl", "VilleCl"],
    "DEFCLASSES.txt": ["NumVol", "Classe", "CoeffPrix"],
    "PILOTES.txt": ["NumPil", "NomPil", "NaisPil", "VillePil"],
    "RESERVATIONS.txt": ["NumCl", "NumVol", "Classe", "NbPlaces"],
    "VOLS.txt": ["NumVol", "VilleD", "VilleA", "DateD", "HD", "DateA", "HA", "NumPil", "NumAv"],
  }
  
  # Lire les fichiers .txt et les stocker dans un dictionnaire
  dictAllJson = {}
  for fileName in os.listdir("src/libs/db/txt"):
    if fileName.endswith(".txt"):
      file_path = os.path.join("src/libs/txt", fileName)
      fields = tableCorespondance[fileName]
  
      # Vérifier si la première colonne est une clé unique
      if fileName in ["DEFCLASSES.txt", "RESERVATIONS.txt"]:
        # Stocker les données dans une liste pour ces fichiers
        dictAllJson[fileName] = []
        with open(file_path, 'r') as fh:
          for line in fh:
            description = list(line.strip().split("\t"))
            if len(description) < len(fields):
              # Gérer les lignes avec des colonnes manquantes
              continue
            data_entry = {}
            for i, categorie in enumerate(fields):
              data_entry[categorie] = description[i]
            dictAllJson[fileName].append(data_entry)
      else:
        # Utiliser un dictionnaire avec la première colonne comme clé
        dictAllJson[fileName] = {}
        with open(file_path, 'r') as fh:
          for line in fh:
            description = list(line.strip().split("\t"))
            if len(description) < len(fields):
              # Gérer les lignes avec des colonnes manquantes
              continue
            key = description[0]
            data_entry = {}
            for i, categorie in enumerate(fields[1:], start=1):
              data_entry[categorie] = description[i]
            dictAllJson[fileName][key] = data_entry
  
  # Construire la liste des vols
  vols_list = []
  for vol_id, vol_data in dictAllJson["VOLS.txt"].items():
    vol_dict = {"_id": vol_id}
  
    # Copier les champs du vol
    vol_fields_mapping = {
      "VilleD": "VilleD",
      "VilleA": "VilleA",
      "DateD": "DateD",
      "HD": "HD",
      "DateA": "DateA",
      "HA": "HA",
    }
    for src_field, dest_field in vol_fields_mapping.items():
      vol_dict[dest_field] = vol_data.get(src_field, "")
  
    # Ajouter l'avion
    num_av = vol_data.get("NumAv", "")
    avion_data = dictAllJson["AVIONS.txt"].get(num_av, {})
    vol_dict["Avion"] = {
      "NumAv": num_av,
      "NomAv": avion_data.get("NomAv", ""),
      "CapAv": avion_data.get("CapAv", ""),
      "VilleAv": avion_data.get("VilleAv", ""),
    }
  
    # Ajouter le pilote
    num_pil = vol_data.get("NumPil", "")
    pilote_data = dictAllJson["PILOTES.txt"].get(num_pil, {})
    vol_dict["Pilote"] = {
      "NumPil": num_pil,
      "NomPil": pilote_data.get("NomPil", ""),
      "NaisPil": pilote_data.get("NaisPil", ""),
      "VillePil": pilote_data.get("VillePil", ""),
    }
  
    # Ajouter les classes
    vol_dict["Classes"] = []
    for classe in dictAllJson["DEFCLASSES.txt"]:
      if classe.get("NumVol") == vol_id:
        vol_dict["Classes"].append({
          "Classe": classe.get("Classe", ""),
          "CoeffPrix": classe.get("CoeffPrix", ""),
        })
  
    vols_list.append(vol_dict)
  
  # Construire la liste des clients
  clients_list = []
  for client_id, client_data in dictAllJson["CLIENTS.txt"].items():
    client_dict = {
      "_id": client_id,
      "NomCl": client_data.get("NomCl", ""),
      "Adresse": {
        "NumRueCl": client_data.get("NumRueCl", ""),
        "NomRueCl": client_data.get("NomRueCl", ""),
        "CodePosteCl": client_data.get("CodePosteCl", ""),
        "VilleCl": client_data.get("VilleCl", ""),
      },
      "Email": "",      # Champs supplémentaires à remplir si nécessaire
      "Telephone": "",  # Champs supplémentaires à remplir si nécessaire
    }
    clients_list.append(client_dict)
  
  # Construire la liste des réservations
  reservations_list = []
  reservation_id_counter = 1
  for reservation_data in dictAllJson["RESERVATIONS.txt"]:
    reservation_dict = {
      "_id": f"R{str(reservation_id_counter).zfill(3)}",
      "VolId": reservation_data.get("NumVol", ""),
      "ClientId": reservation_data.get("NumCl", ""),
      "NbPlaces": reservation_data.get("NbPlaces", ""),
      "Classe": reservation_data.get("Classe", ""),
    }
    reservations_list.append(reservation_dict)
    reservation_id_counter += 1
  
  # Écrire les données dans des fichiers JSON séparés
  with open("src/libs/db/json/vols.json", "w") as f:
    json.dump(vols_list, f, indent=2, ensure_ascii=False)
  
  with open("src/libs/db/json/clients.json", "w") as f:
    json.dump(clients_list, f, indent=2, ensure_ascii=False)
  
  with open("src/libs/db/json/reservations.json", "w") as f:
    json.dump(reservations_list, f, indent=2, ensure_ascii=False)
  
  print("Données extraites avec succès.")
\end{lstlisting}

\section{Insertion des documents dans Redis}
\label{ann:redis_insert}
\begin{lstlisting}[language=Python, caption=Script d'insertion des documents dans Redis, label=lst:redis_insert]
  import json
  import sys, os

  # Définir le chemin 
  sys.path.append(os.path.abspath(os.path.join(os.path.dirname(__file__), '..', '..')))

  # Définir le chemin vers le dossier contenant les fichiers JSON
  SRC = os.path.abspath(os.path.join(os.path.dirname(__file__), '..', '..'))
  DATA_DIR = os.path.join(SRC, 'libs', 'db', 'json')

  from config.connectionRedis import connect_redis as connect

  # Charger les données JSON
  with open(os.path.join(DATA_DIR, 'vols.json'), 'r', encoding='utf-8') as f: 
  vols = json.load(f)
  with open(os.path.join(DATA_DIR, 'clients.json'), 'r', encoding='utf-8') as f:
    clients = json.load(f)
  with open(os.path.join(DATA_DIR, 'reservations.json'), 'r', encoding='utf-8') as f:
    reservations = json.load(f)

  r = connect()

  def clear_redis_database():
    r.flushdb()
    
  clear_redis_database()

  # Insérer les vols dans Redis
  for vol in vols:
    vol_id = vol['_id']
    # Stocker le vol sous la clé 'vol:{vol_id}'
    r.set(f'vol:{vol_id}', json.dumps(vol))

  # Insérer les clients dans Redis
  for client in clients:
    client_id = client['_id']
    # Stocker le client sous la clé 'client:{client_id}'
    r.set(f'client:{client_id}', json.dumps(client))

  # Insérer les réservations dans Redis
  for reservation in reservations:
    reservation_id = reservation['_id']
    # Stocker la réservation sous la clé 'reservation:{reservation_id}'
    r.set(f'reservation:{reservation_id}', json.dumps(reservation))

  print("Données insérées avec succès dans Redis.")
\end{lstlisting}

\subsection{Insertion des documents dans MongoDB}
\label{ann:mongo_insert}
\begin{lstlisting}[language=Python, caption=Script d'insertion des documents dans MongoDB, label=lst:mongo_insert]
  import sys, os, json

  sys.path.append(os.path.abspath(os.path.join(os.path.dirname(__file__), '..', '..')))

  from config.connectionMongo import connect_mongodb as connect

  # Définir le chemin vers le dossier contenant les fichiers JSON
  SRC = os.path.abspath(os.path.join(os.path.dirname(__file__), '..', '..'))
  DATA_DIR = os.path.join(SRC, 'libs', 'DAL', 'db', 'json')

  # Charger les données JSON
  with open(os.path.join(DATA_DIR, 'vols.json'), 'r', encoding='utf-8') as f: 
    vols = json.load(f)
  with open(os.path.join(DATA_DIR, 'clients.json'), 'r', encoding='utf-8') as f: 
    clients = json.load(f)
  with open(os.path.join(DATA_DIR, 'reservations.json'), 'r', encoding='utf-8') as f: 
    reservations = json.load(f)

  # Se connecter à MongoDB
  client = connect()

  db = client.get_database('BookingSystem')

  # Créer ou obtenir les collections
  vols_collection = db.get_collection('vols')
  clients_collection = db.get_collection('clients')
  reservations_collection = db.get_collection('reservations')

  # Effacer les collections existantes pour éviter les doublons lors de ré-exécutions
  vols_collection.delete_many({})
  clients_collection.delete_many({})
  reservations_collection.delete_many({})

  # Insérer les documents JSON dans chaque collection respective
  vols_collection.insert_many(vols)
  clients_collection.insert_many(clients)
  reservations_collection.insert_many(reservations)

  print("Les données ont été insérées avec succès dans MongoDB.")
\end{lstlisting}

\section{Résultats des requêtes}
\label{ann:results_queries}
\begin{lstlisting}[language=Python, caption=Résultats des requêtes, label=lst:results_queries]
  [MongoDB] Function 'get_arrival_cities' executed in 0.024853 seconds
  Liste des villes d'arrivée :
  Marseille: 62
  Amsterdam: 41
  NewYork: 30
  Nice: 29
  Pekin: 28
  Paris: 9
  
  [MongoDB] Function 'get_arrival_city_by_id' executed in 0.002618 seconds
  Ville d'arrivée du vol V519 : NewYork
  
  Liste des pilotes :
  [MongoDB] Function 'get_pilotes' executed in 0.002133 seconds
  Delacroix
  Delalande
  Dubois
  Dumas
  Duparc
  Duval
  Leblanc
  Ledru
  Legrand
  [MongoDB] Function 'get_pilotes' executed in 0.001526 seconds
  Nombre de pilotes: 9
  Vols au départ de Marseille :
  [MongoDB] Function 'get_vols_by_departure_city' executed in 0.005160 seconds
  V101 -> Amsterdam
  V102 -> Amsterdam
  V103 -> Amsterdam
  V104 -> Amsterdam
  V105 -> Amsterdam
  V106 -> Amsterdam
  V107 -> Amsterdam
  V108 -> Amsterdam
  V109 -> Amsterdam
  V110 -> Amsterdam
  V111 -> Amsterdam
  V112 -> Amsterdam
  V113 -> Amsterdam
  V114 -> Amsterdam
  V115 -> Amsterdam
  V116 -> Amsterdam
  V117 -> Amsterdam
  V118 -> Amsterdam
  V119 -> Amsterdam
  V120 -> Amsterdam
  V180 -> Amsterdam
  V181 -> Amsterdam
  V182 -> Amsterdam
  V183 -> Amsterdam
  V184 -> Amsterdam
  V185 -> Amsterdam
  V186 -> Amsterdam
  V187 -> Amsterdam
  V188 -> Amsterdam
  V189 -> Amsterdam
  V190 -> Amsterdam
  V191 -> Amsterdam
  V192 -> Amsterdam
  V193 -> Amsterdam
  V194 -> Amsterdam
  V195 -> Amsterdam
  V196 -> Amsterdam
  V197 -> Amsterdam
  V198 -> Amsterdam
  V199 -> Amsterdam
  V200 -> Amsterdam
  V350 -> Paris
  V351 -> Paris
  V501 -> NewYork
  V502 -> NewYork
  V503 -> NewYork
  V504 -> NewYork
  V505 -> NewYork
  V506 -> NewYork
  V507 -> NewYork
  V508 -> NewYork
  V509 -> NewYork
  V510 -> NewYork
  V511 -> NewYork
  V512 -> NewYork
  V513 -> NewYork
  V514 -> NewYork
  V515 -> NewYork
  V516 -> NewYork
  V517 -> NewYork
  V518 -> NewYork
  V519 -> NewYork
  V520 -> NewYork
  V521 -> NewYork
  V522 -> NewYork
  V523 -> NewYork
  V524 -> NewYork
  V525 -> NewYork
  V526 -> NewYork
  V527 -> NewYork
  V528 -> NewYork
  V529 -> NewYork
  V530 -> NewYork
  V681 -> Pekin
  V682 -> Pekin
  V683 -> Pekin
  V684 -> Pekin
  V685 -> Pekin
  V686 -> Pekin
  V687 -> Pekin
  V688 -> Pekin
  V689 -> Pekin
  V690 -> Pekin
  V691 -> Pekin
  V692 -> Pekin
  V693 -> Pekin
  V694 -> Pekin
  
  [MongoDB] Function 'get_reservations_by_client' executed in 0.007681 seconds
  Réservations pour le client 1001 :
  Réservation ID: R001, Vol ID: V690, Classe: Business, NbPlaces: 3
  Détails du vol: De Marseille à Pekin le 19/04/07
  
  Réservation ID: R016, Vol ID: V141, Classe: Business, NbPlaces: 3
  Détails du vol: De Amsterdam à Marseille le 1/04/07
  
  Réservation ID: R021, Vol ID: V790, Classe: Touriste, NbPlaces: 1
  Détails du vol: De Metz à Marseille le 20/04/07
  
  Réservation ID: R022, Vol ID: V150, Classe: Touriste, NbPlaces: 1
  Détails du vol: De Amsterdam à Marseille le 10/04/07
  
  Réservation ID: R033, Vol ID: V601, Classe: Economique, NbPlaces: 6
  Détails du vol: De Ajaccio à Marseille le 1/04/07
  
  [MongoDB] Function 'get_clients_by_flight' executed in 0.001317 seconds
  Clients sur le vol V519 :
  [MongoDB] Function 'get_flights_by_pilot' executed in 0.002763 seconds
  Vols opérés par le pilote Leblanc :
  Vol ID: V101, De Marseille à Amsterdam le 1/04/07
  Vol ID: V102, De Marseille à Amsterdam le 2/04/07
  Vol ID: V103, De Marseille à Amsterdam le 3/04/07
  Vol ID: V104, De Marseille à Amsterdam le 4/04/07
  Vol ID: V105, De Marseille à Amsterdam le 5/04/07
  Vol ID: V106, De Marseille à Amsterdam le 6/04/07
  Vol ID: V107, De Marseille à Amsterdam le 7/04/07
  Vol ID: V108, De Marseille à Amsterdam le 8/04/07
  Vol ID: V109, De Marseille à Amsterdam le 9/04/07
  Vol ID: V110, De Marseille à Amsterdam le 10/04/07
  Vol ID: V111, De Marseille à Amsterdam le 11/04/07
  Vol ID: V112, De Marseille à Amsterdam le 12/04/07
  Vol ID: V113, De Marseille à Amsterdam le 13/04/07
  Vol ID: V114, De Marseille à Amsterdam le 14/04/07
  Vol ID: V115, De Marseille à Amsterdam le 15/04/07
  Vol ID: V116, De Marseille à Amsterdam le 16/04/07
  Vol ID: V117, De Marseille à Amsterdam le 17/04/07
  Vol ID: V118, De Marseille à Amsterdam le 18/04/07
  Vol ID: V119, De Marseille à Amsterdam le 19/04/07
  Vol ID: V120, De Marseille à Amsterdam le 20/04/07
  Vol ID: V141, De Amsterdam à Marseille le 1/04/07
  Vol ID: V142, De Amsterdam à Marseille le 2/04/07
  Vol ID: V143, De Amsterdam à Marseille le 3/04/07
  Vol ID: V144, De Amsterdam à Marseille le 4/04/07
  Vol ID: V145, De Amsterdam à Marseille le 5/04/07
  Vol ID: V146, De Amsterdam à Marseille le 6/04/07
  Vol ID: V147, De Amsterdam à Marseille le 7/04/07
  Vol ID: V148, De Amsterdam à Marseille le 8/04/07
  Vol ID: V149, De Amsterdam à Marseille le 9/04/07
  Vol ID: V150, De Amsterdam à Marseille le 10/04/07
  Vol ID: V151, De Amsterdam à Marseille le 11/04/07
  Vol ID: V152, De Amsterdam à Marseille le 12/04/07
  Vol ID: V153, De Amsterdam à Marseille le 13/04/07
  Vol ID: V154, De Amsterdam à Marseille le 14/04/07
  Vol ID: V155, De Amsterdam à Marseille le 15/04/07
  Vol ID: V156, De Amsterdam à Marseille le 16/04/07
  Vol ID: V157, De Amsterdam à Marseille le 17/04/07
  Vol ID: V180, De Marseille à Amsterdam le 10/04/07
  Vol ID: V181, De Marseille à Amsterdam le 11/04/07
  Vol ID: V182, De Marseille à Amsterdam le 12/04/07
  Vol ID: V183, De Marseille à Amsterdam le 13/04/07
  Vol ID: V184, De Marseille à Amsterdam le 14/04/07
  Vol ID: V185, De Marseille à Amsterdam le 15/04/07
  Vol ID: V186, De Marseille à Amsterdam le 16/04/07
  Vol ID: V187, De Marseille à Amsterdam le 17/04/07
  Vol ID: V188, De Marseille à Amsterdam le 18/04/07
  Vol ID: V189, De Marseille à Amsterdam le 19/04/07
  Vol ID: V190, De Marseille à Amsterdam le 20/04/07
  Vol ID: V191, De Marseille à Amsterdam le 21/04/07
  Vol ID: V192, De Marseille à Amsterdam le 22/04/07
  Vol ID: V193, De Marseille à Amsterdam le 23/04/07
  Vol ID: V194, De Marseille à Amsterdam le 24/04/07
  Vol ID: V195, De Marseille à Amsterdam le 25/04/07
  Vol ID: V196, De Marseille à Amsterdam le 26/04/07
  Vol ID: V197, De Marseille à Amsterdam le 27/04/07
  Vol ID: V198, De Marseille à Amsterdam le 28/04/07
  Vol ID: V199, De Marseille à Amsterdam le 29/04/07
  Vol ID: V200, De Marseille à Amsterdam le 30/04/07
  
  [MongoDB] Function 'get_clients_by_pilot' executed in 0.011204 seconds
  Clients ayant des réservations sur les vols du pilote Leblanc :
  Client ID: 1031, Nom: Bohr, Vol ID: V101, NbPlaces: 2, Classe: Business
  
  Client ID: 1027, Nom: Lorentz, Vol ID: V101, NbPlaces: 2, Classe: Business
  
  Client ID: 1031, Nom: Bohr, Vol ID: V101, NbPlaces: 5, Classe: Touriste
  
  Client ID: 1033, Nom: Dirac, Vol ID: V101, NbPlaces: 7, Classe: Touriste
  
  Client ID: 1028, Nom: Lenard, Vol ID: V101, NbPlaces: 6, Classe: Touriste
  
  Client ID: 1021, Nom: Perse, Vol ID: V101, NbPlaces: 6, Classe: Touriste
  
  Client ID: 1002, Nom: Leblanc, Vol ID: V101, NbPlaces: 5, Classe: Touriste
  
  Client ID: 1029, Nom: Planck, Vol ID: V101, NbPlaces: 7, Classe: Economique
  
  Client ID: 1015, Nom: Rolland, Vol ID: V101, NbPlaces: 1, Classe: Economique
  
  Client ID: 1006, Nom: Grignard, Vol ID: V101, NbPlaces: 4, Classe: Economique
  
  Client ID: 1001, Nom: Becquerel, Vol ID: V141, NbPlaces: 3, Classe: Business
  
  Client ID: 1008, Nom: Joliot, Vol ID: V141, NbPlaces: 2, Classe: Touriste
  
  Client ID: 1011, Nom: Richet, Vol ID: V141, NbPlaces: 8, Classe: Economique
  
  Client ID: 1020, Nom: Gide, Vol ID: V150, NbPlaces: 1, Classe: Business
  
  Client ID: 1027, Nom: Lorentz, Vol ID: V150, NbPlaces: 1, Classe: Business
  
  Client ID: 1001, Nom: Becquerel, Vol ID: V150, NbPlaces: 1, Classe: Touriste
  
  Client ID: 1029, Nom: Planck, Vol ID: V150, NbPlaces: 7, Classe: Economique
\end{lstlisting}

\section{Profiling des Requêtes}
\label{ann:profiling}
\begin{lstlisting}[language=Python, caption=Profiling des requêtes, label=lst:profiling]
import time
from functools import wraps
import csv

# Global list to store performance results
performance_results = []

def timing_decorator(method_group = None):
  """
  A decorator to measure execution time of methods and log the results.

  Args:
    method_group (str): The group or class name (e.g., 'MongoDB' or 'JSON').

  Returns:
    function: The decorated function with added timing code.
  """
  def decorator(func):
    @wraps(func)
    def wrapper(*args, **kwargs):
      nonlocal method_group
      if method_group is None and args:
        self = args[0]
        method_group = getattr(self, 'method_group', 'Undefined')

      start_time = time.perf_counter()

      result = func(*args, **kwargs)

      end_time = time.perf_counter()

      execution_time = end_time - start_time

      performance_results.append({
        'method_group': method_group,
        'function_name': func.__name__,
        'execution_time': execution_time
      })
      # Print the result
      print(f"[{method_group}] Function '{func.__name__}' executed in {execution_time:.6f} seconds")
      # Return the original function's result
      return result
    return wrapper
  return decorator
\end{lstlisting}

\section{Dépôt Git}
\label{ann:git_repo}
Vous trouverez le dépôt Git de ce projet sur GitHub: \href{https://github.com/sessaadouni/BookingSystem-app}{GitHub - BookingSystem-app}.

Pour pouvoir tester le projet vous pourrez suivre les étapes mises dans le fichier \href{https://github.com/sessaadouni/BookingSystem-app/blob/master/README.md}{\texttt{README.md}}.