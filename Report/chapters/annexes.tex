% Annexes
\appendix
\chapter{Annexes}

\section{Configuration Docker pour Redis}
\label{ann:docker_redis}
\begin{verbatim}
redis:
  build:
    context: ./.docker/redis/.
    dockerfile: Dockerfile.redis
  ports:
    - "${REDIS_PORT}:${REDIS_PORT}"
  volumes:
    - ./.docker-data/redis/data:/data
    - ./.docker-data/redis/logs:/var/log/redis
  restart: always
  env_file:
    - ./.docker/redis/.env.redis
  environment:
    - REDIS_ARGS=--requirepass ${REDIS_PASSWORD}
  healthcheck:
    test: ["CMD", "redis-cli", "-a", "${REDIS_PASSWORD}", "ping"]
    interval: 30s
    timeout: 10s
    retries: 5
\end{verbatim}

\section{Configuration Docker pour MongoDB}
\label{ann:docker_mongo}
\begin{verbatim}
mongo:
  build:
    context: ./.docker/mongo/
    dockerfile: Dockerfile.mongo
  ports:
    - 27017:27017 # Expose le port 27017
    - 27018:27018 # Expose le port 27018
    - 27019:27019 # Expose le port 27019
  restart: always # Redémarre toujours en cas de crash
  volumes:
    - ./.docker-data/mongo/data:/data/db # Monte le volume de données MongoDB
  env_file:
    - ./.docker/mongo/.env.mongo # Fichier d'environnement pour MongoDB
  healthcheck: # Vérifie l'état de santé de MongoDB
    test: echo 'db.runCommand("ping").ok' | mongosh mongo:27017/test --quiet
    interval: 30s # Intervalle entre les vérifications
    timeout: 10s # Délai d'expiration de la commande
    retries: 5 # Nombre de tentatives avant de considérer le service comme défaillant
    start_period: 20s # Temps avant le démarrage des vérifications
  networks:
    dev:
      aliases:
        - localhost # Alias pour le service MongoDB dans le réseau Docker
\end{verbatim}

\section{Configuration Docker pour Python}
\label{ann:docker_poetry}
\begin{verbatim}
poetry:
  build:
    context: ./.docker/poetry/.
    dockerfile: Dockerfile.poetry
  stdin_open: true
  tty: true
  volumes:
    - .:/workspace
  working_dir: /workspace
  environment:
    - POETRY_VIRTUALENVS_IN_PROJECT=true
  command: [ "/bin/sh" ]
\end{verbatim}

\section{Création des Documents JSON}
\label{ann:code_json}
\begin{verbatim}
  import json
  import os
  
  # Table de correspondance des colonnes
  tableCorespondance = {
    "AVIONS.txt": ["NumAv", "NomAv", "CapAv", "VilleAv"],
    "CLIENTS.txt": ["NumCl", "NomCl", "NumRueCl", "NomRueCl", "CodePosteCl", "VilleCl"],
    "DEFCLASSES.txt": ["NumVol", "Classe", "CoeffPrix"],
    "PILOTES.txt": ["NumPil", "NomPil", "NaisPil", "VillePil"],
    "RESERVATIONS.txt": ["NumCl", "NumVol", "Classe", "NbPlaces"],
    "VOLS.txt": ["NumVol", "VilleD", "VilleA", "DateD", "HD", "DateA", "HA", "NumPil", "NumAv"],
  }
  
  # Lire les fichiers .txt et les stocker dans un dictionnaire
  dictAllJson = {}
  for fileName in os.listdir("src/libs/db/txt"):
    if fileName.endswith(".txt"):
      file_path = os.path.join("src/libs/txt", fileName)
      fields = tableCorespondance[fileName]
  
      # Vérifier si la première colonne est une clé unique
      if fileName in ["DEFCLASSES.txt", "RESERVATIONS.txt"]:
        # Stocker les données dans une liste pour ces fichiers
        dictAllJson[fileName] = []
        with open(file_path, 'r') as fh:
          for line in fh:
            description = list(line.strip().split("\t"))
            if len(description) < len(fields):
              # Gérer les lignes avec des colonnes manquantes
              continue
            data_entry = {}
            for i, categorie in enumerate(fields):
              data_entry[categorie] = description[i]
            dictAllJson[fileName].append(data_entry)
      else:
        # Utiliser un dictionnaire avec la première colonne comme clé
        dictAllJson[fileName] = {}
        with open(file_path, 'r') as fh:
          for line in fh:
            description = list(line.strip().split("\t"))
            if len(description) < len(fields):
              # Gérer les lignes avec des colonnes manquantes
              continue
            key = description[0]
            data_entry = {}
            for i, categorie in enumerate(fields[1:], start=1):
              data_entry[categorie] = description[i]
            dictAllJson[fileName][key] = data_entry
  
  # Construire la liste des vols
  vols_list = []
  for vol_id, vol_data in dictAllJson["VOLS.txt"].items():
    vol_dict = {"_id": vol_id}
  
    # Copier les champs du vol
    vol_fields_mapping = {
      "VilleD": "VilleD",
      "VilleA": "VilleA",
      "DateD": "DateD",
      "HD": "HD",
      "DateA": "DateA",
      "HA": "HA",
    }
    for src_field, dest_field in vol_fields_mapping.items():
      vol_dict[dest_field] = vol_data.get(src_field, "")
  
    # Ajouter l'avion
    num_av = vol_data.get("NumAv", "")
    avion_data = dictAllJson["AVIONS.txt"].get(num_av, {})
    vol_dict["Avion"] = {
      "NumAv": num_av,
      "NomAv": avion_data.get("NomAv", ""),
      "CapAv": avion_data.get("CapAv", ""),
      "VilleAv": avion_data.get("VilleAv", ""),
    }
  
    # Ajouter le pilote
    num_pil = vol_data.get("NumPil", "")
    pilote_data = dictAllJson["PILOTES.txt"].get(num_pil, {})
    vol_dict["Pilote"] = {
      "NumPil": num_pil,
      "NomPil": pilote_data.get("NomPil", ""),
      "NaisPil": pilote_data.get("NaisPil", ""),
      "VillePil": pilote_data.get("VillePil", ""),
    }
  
    # Ajouter les classes
    vol_dict["Classes"] = []
    for classe in dictAllJson["DEFCLASSES.txt"]:
      if classe.get("NumVol") == vol_id:
        vol_dict["Classes"].append({
          "Classe": classe.get("Classe", ""),
          "CoeffPrix": classe.get("CoeffPrix", ""),
        })
  
    vols_list.append(vol_dict)
  
  # Construire la liste des clients
  clients_list = []
  for client_id, client_data in dictAllJson["CLIENTS.txt"].items():
    client_dict = {
      "_id": client_id,
      "NomCl": client_data.get("NomCl", ""),
      "Adresse": {
        "NumRueCl": client_data.get("NumRueCl", ""),
        "NomRueCl": client_data.get("NomRueCl", ""),
        "CodePosteCl": client_data.get("CodePosteCl", ""),
        "VilleCl": client_data.get("VilleCl", ""),
      },
      "Email": "",      # Champs supplémentaires à remplir si nécessaire
      "Telephone": "",  # Champs supplémentaires à remplir si nécessaire
    }
    clients_list.append(client_dict)
  
  # Construire la liste des réservations
  reservations_list = []
  reservation_id_counter = 1
  for reservation_data in dictAllJson["RESERVATIONS.txt"]:
    reservation_dict = {
      "_id": f"R{str(reservation_id_counter).zfill(3)}",
      "VolId": reservation_data.get("NumVol", ""),
      "ClientId": reservation_data.get("NumCl", ""),
      "NbPlaces": reservation_data.get("NbPlaces", ""),
      "Classe": reservation_data.get("Classe", ""),
    }
    reservations_list.append(reservation_dict)
    reservation_id_counter += 1
  
  # Écrire les données dans des fichiers JSON séparés
  with open("src/libs/db/json/vols.json", "w") as f:
    json.dump(vols_list, f, indent=2, ensure_ascii=False)
  
  with open("src/libs/db/json/clients.json", "w") as f:
    json.dump(clients_list, f, indent=2, ensure_ascii=False)
  
  with open("src/libs/db/json/reservations.json", "w") as f:
    json.dump(reservations_list, f, indent=2, ensure_ascii=False)
  
  print("Données extraites avec succès.")
\end{verbatim}

\section{Insertion des documents dans Redis}
\label{ann:redis_insert}
\begin{verbatim}
  import json
  import sys, os

  # Définir le chemin 
  sys.path.append(os.path.abspath(os.path.join(os.path.dirname(__file__), '..', '..')))

  # Définir le chemin vers le dossier contenant les fichiers JSON
  SRC = os.path.abspath(os.path.join(os.path.dirname(__file__), '..', '..'))
  DATA_DIR = os.path.join(SRC, 'libs', 'db', 'json')

  from config.connectionRedis import connect_redis as connect

  # Charger les données JSON
  with open(os.path.join(DATA_DIR, 'vols.json'), 'r', encoding='utf-8') as f: vols = json.load(f)
  with open(os.path.join(DATA_DIR, 'clients.json'), 'r', encoding='utf-8') as f:
    clients = json.load(f)
  with open(os.path.join(DATA_DIR, 'reservations.json'), 'r', encoding='utf-8') as f:
    reservations = json.load(f)

  r = connect()

  def clear_redis_database():
    r.flushdb()
    
  clear_redis_database()

  # Insérer les vols dans Redis
  for vol in vols:
    vol_id = vol['_id']
    # Stocker le vol sous la clé 'vol:{vol_id}'
    r.set(f'vol:{vol_id}', json.dumps(vol))

  # Insérer les clients dans Redis
  for client in clients:
    client_id = client['_id']
    # Stocker le client sous la clé 'client:{client_id}'
    r.set(f'client:{client_id}', json.dumps(client))

  # Insérer les réservations dans Redis
  for reservation in reservations:
    reservation_id = reservation['_id']
    # Stocker la réservation sous la clé 'reservation:{reservation_id}'
    r.set(f'reservation:{reservation_id}', json.dumps(reservation))

  print("Données insérées avec succès dans Redis.")
\end{verbatim}

\subsection{Insertion des documents dans MongoDB}
\label{ann:mongo_insert}
\begin{verbatim}
  import sys, os, json

  sys.path.append(os.path.abspath(os.path.join(os.path.dirname(__file__), '..', '..')))

  from config.connectionMongo import connect_mongodb as connect

  # Définir le chemin vers le dossier contenant les fichiers JSON
  SRC = os.path.abspath(os.path.join(os.path.dirname(__file__), '..', '..'))
  DATA_DIR = os.path.join(SRC, 'libs', 'DAL', 'db', 'json')

  # Charger les données JSON
  with open(os.path.join(DATA_DIR, 'vols.json'), 'r', encoding='utf-8') as f: vols = json.load(f)
  with open(os.path.join(DATA_DIR, 'clients.json'), 'r', encoding='utf-8') as f: clients = json.load(f)
  with open(os.path.join(DATA_DIR, 'reservations.json'), 'r', encoding='utf-8') as f: reservations = json.load(f)

  # Se connecter à MongoDB
  client = connect()

  db = client.get_database('BookingSystem')

  # Créer ou obtenir les collections
  vols_collection = db.get_collection('vols')
  clients_collection = db.get_collection('clients')
  reservations_collection = db.get_collection('reservations')

  # Effacer les collections existantes pour éviter les doublons lors de ré-exécutions
  vols_collection.delete_many({})
  clients_collection.delete_many({})
  reservations_collection.delete_many({})

  # Insérer les documents JSON dans chaque collection respective
  vols_collection.insert_many(vols)
  clients_collection.insert_many(clients)
  reservations_collection.insert_many(reservations)

  print("Les données ont été insérées avec succès dans MongoDB.")
\end{verbatim}

\section{Profiling des Requêtes}
\label{ann:profiling}
\begin{verbatim}
# Détails de la performance des requêtes dans chaque base
\end{verbatim}

\section{Dépôt Git}
\label{ann:git_repo}
Vous trouverez le dépôt Git de ce projet sur GitHub: \url{https://github.com/}.

Pour pouvoir tester le projet vous pourrez suivre les étapes mises dans le fichier \texttt{README.md}.