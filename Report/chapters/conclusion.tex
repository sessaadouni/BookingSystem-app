\chapter{Conclusion}

La migration de bases de données relationnelles SQL vers des solutions NoSQL, en utilisant notamment Redis et MongoDB, s'est avérée être une stratégie efficace pour répondre aux besoins croissants de flexibilité et de performance dans la gestion des données. Ce projet a mis en évidence les avantages significatifs qu'offrent ces technologies pour le stockage et la manipulation de données semi-structurées.

Les principaux résultats obtenus sont les suivants :

\begin{itemize}
  \item \textbf{Optimisation des Performances :} Les tests de performance ont démontré que Redis offre des temps de réponse extrêmement rapides pour les opérations de lecture simples sur de petits volumes de données, grâce à son stockage en mémoire. MongoDB, quant à lui, assure une meilleure performance et une scalabilité accrue pour les requêtes complexes et les grands volumes de données, grâce à son système de gestion de documents JSON natif.

  \item \textbf{Flexibilité Accrue :} La transition vers des bases NoSQL a permis de dénormaliser les données et de les structurer de manière plus flexible. Cette approche facilite l'adaptation aux changements des modèles de données et réduit la complexité des opérations de manipulation des données.

  \item \textbf{Simplification des Requêtes :} En regroupant les informations auparavant réparties dans plusieurs tables SQL au sein de documents JSON unifiés, nous avons simplifié les requêtes nécessaires pour accéder aux données, réduisant ainsi le nombre de jointures et améliorant les temps de réponse.

  \item \textbf{Évolution de l'Architecture :} L'utilisation de Docker pour orchestrer les différents services a permis de mettre en place une architecture modulaire et facilement déployable. Cette approche facilite la gestion des environnements de développement et de production, tout en assurant une isolation des services.

  \item \textbf{Comparaison Objectivée des Bases de Données :} L'analyse comparative entre Redis et MongoDB a mis en lumière leurs forces et faiblesses respectives. Redis est idéal pour des applications nécessitant des accès ultra-rapides à des données volatiles, tandis que MongoDB est plus adapté pour des applications nécessitant une persistance des données et des requêtes complexes.

\end{itemize}

Ces résultats confirment que l'utilisation combinée de Redis et MongoDB peut offrir une solution robuste et performante pour des applications modernes, où la flexibilité et la performance sont essentielles.

\section*{Limites du Projet}

Malgré les avantages constatés, certaines limitations ont été identifiées :

\begin{itemize}
  \item \textbf{Gestion des Transactions :} Les bases NoSQL ne supportent pas les transactions complexes de la même manière que les bases SQL, ce qui peut poser des défis pour garantir l'intégrité des données dans certaines applications critiques.

  \item \textbf{Consistance des Données :} La flexibilité des schémas dans les bases NoSQL peut entraîner des incohérences si les données ne sont pas correctement validées au niveau de l'application.

  \item \textbf{Courbe d'Apprentissage :} La migration vers NoSQL nécessite une adaptation des compétences des développeurs et des administrateurs de bases de données, ainsi qu'une compréhension approfondie des nouveaux paradigmes de modélisation des données.

\end{itemize}

\section*{Perspectives Futures}

Pour prolonger ce travail, plusieurs axes peuvent être explorés :

\begin{itemize}
  \item \textbf{Mise en Place de Mécanismes de Validation :} Intégrer des schémas de validation au niveau de l'application ou utiliser des fonctionnalités offertes par les bases NoSQL pour assurer la cohérence des données.

  \item \textbf{Évaluation d'Autres Bases NoSQL :} Étudier l'intégration de bases telles que Cassandra, CouchDB ou Elasticsearch pour comparer leurs performances et leurs fonctionnalités avec celles de Redis et MongoDB.
  
  \item \textbf{Évaluation d'Autres Bases comme le NewSQL :} 

  \item \textbf{Optimisation des Performances :} Mettre en œuvre des mécanismes de mise en cache avancés, comme Redis Cache, pour améliorer encore les temps de réponse des applications.

  \item \textbf{Sécurité et Gestion des Accès :} Explorer les options de sécurité offertes par les bases NoSQL, y compris l'authentification, l'autorisation et le chiffrement des données.

  \item \textbf{Scalabilité Horizontale :} Tester la mise en place de clusters Redis et MongoDB pour évaluer les performances en environnement distribué et la tolérance aux pannes.

  \item \textbf{Automatisation du Déploiement :} Utiliser des outils d'orchestration tels que Kubernetes pour automatiser le déploiement et la gestion des conteneurs Docker dans des environnements de production.

\end{itemize}

\section*{Mot de la Fin}

Ce projet a permis de démontrer concrètement les bénéfices de la migration vers des bases de données NoSQL dans un contexte où la flexibilité et la performance sont primordiales. Les technologies étudiées offrent des perspectives prometteuses pour le développement d'applications modernes capables de gérer efficacement des volumes de données en constante augmentation.

En conclusion, le choix entre les bases de données relationnelles et les bases NoSQL doit être guidé par les besoins spécifiques de l'application, en tenant compte des contraintes de performance, de flexibilité et de scalabilité. La combinaison de plusieurs technologies, comme Redis et MongoDB, peut offrir une solution hybride adaptée aux exigences complexes des systèmes d'information actuels.

