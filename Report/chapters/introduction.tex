\chapter{Introduction}

Dans le cadre de ce projet, nous avons entrepris la transformation d'une base de données relationnelle (SQL) classique en une architecture NoSQL, une démarche qui répond aux exigences modernes de flexibilité, d'évolutivité, et de gestion optimisée des données semi-structurées. Face à l’augmentation du volume de données et aux besoins croissants d’agilité dans les traitements, le modèle SQL a montré des limites, notamment en termes de performance et de rigidité. C’est pour cela que nous avons décidé de mettre en place une solution NoSQL qui permet d’adapter la structure des données aux besoins actuels.

Le choix d'une dénormalisation en format JSON, un processus permettant de simplifier et d’optimiser la manipulation des données, est au cœur de cette transformation. En effet, ce format est particulièrement bien adapté aux bases de données NoSQL, car il facilite l’agrégation des informations liées, élimine les jointures coûteuses, et permet une gestion simplifiée des données sans contrainte de schéma rigide.

Dans cette étude, nous avons utilisé deux bases NoSQL majeures:\@ Redis et MongoDB.\@ Redis, avec sa capacité de stockage en mémoire, offre des performances optimisées pour les lectures rapides et les données temporaires, tandis que MongoDB se distingue par sa persistance de données et sa capacité à gérer des structures JSON complexes. Ce rapport détaille les étapes de la migration, les choix techniques, et les avantages obtenus en termes de performance et de flexibilité, illustrant comment ces technologies peuvent répondre aux besoins métiers modernes.

\vspace{5cm}

\section*{Résumé}
\begin{center}
Ce rapport décrit le processus de migration d'une base de données SQL vers un environnement NoSQL, en adoptant une approche de dénormalisation au format JSON.\@ En intégrant Redis et MongoDB, nous démontrons les avantages d'une architecture NoSQL pour le stockage et la manipulation de données semi-structurées, avec des gains en flexibilité et en performance. La combinaison de ces technologies permet d'adapter la gestion des données aux exigences actuelles, en simplifiant les traitements tout en assurant une évolutivité pour des volumes de données plus élevés.
\end{center}
